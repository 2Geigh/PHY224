\documentclass[12pt]{article}
\usepackage{graphicx}
%\usepackage{comment}
%\usepackage{wrapfig}
\usepackage{hyperref}
\setlength{\textwidth}{6.5in}
\setlength{\textheight}{9.9in}
\setlength{\oddsidemargin}{0.0in}
\setlength{\evensidemargin}{0.0in}
\setlength{\topmargin}{-2.5 cm}
\setlength{\parskip}{0.5\baselineskip} % space between paragraphs
\setlength{\parindent}{0 cm} % increase this if you like paragraphs to be indented
\author{Firstname Lastname} %PUT YOUR NAME HERE

\pagestyle{empty}

\newcommand{\e}{\mathrm{e}}

\begin{document}
\begin{center}
\bf{\LARGE 
PHY224 TeX Template/Guide %REPLACE THIS WITH YOUR REPORT TITLE
}
\end{center}

\section{Introduction}

Your introduction goes here. It should include equations. Here's an equation. Making nice equations is the single biggest reason to use TeX.
\begin{equation} 
	\label{DampedHarmonic}
	\theta(t) = \theta_0 ~ \e^{-t/\tau} \cos{\left(2\pi \frac{t}{T}+\phi_0 \right)}
\end{equation}
The label command lets you refer to it; typing \verb!Equation \ref{DampedHarmonic}! will give you Equation \ref{DampedHarmonic}. The equation number will update based on what order your equations are in, so there's no need to renumber all your equations and equation references every time you rearrange them!

Note that you can also put equations in a paragraph using math mode, with dollar signs (\verb!$y(x)=mx+b$! to get $y(x)=mx+b$). You cannot reference them though. Variables ($x$, $y$, etc.) should be typed in math mode, i.e. with dollar signs around them. You'll also need the dollar signs to type Greek letters (e.g. \verb!\pi! to get $\pi$, \verb!\phi! to get $\phi$, etc.).

Here are some other equations:

$T = T_0(1  + B \theta_0 + C \theta_0^2 + \ldots )$ \\
$Q = \pi \frac{\tau}{T}$ \\
$T =k\, L^n$ \\
$T = 2 \sqrt{L}$

Take a look at the code in the .tex file to see how the fractions, subscripts, etc. are made.

Note that the \verb!\\! command forces a new line
(simply pressing enter won't work).
The \verb!~! symbol forces a space, which ~ can ~ ~ be ~ ~ ~ handy.

For a list of mathematical symbols in LaTeX, check out \url{https://oeis.org/wiki/List_of_LaTeX_mathematical_symbols}. \\

\section{Methods and Procedures}

Write your methods and procedures here.

\section{Results and Analysis}

Section for results and analysis.

\subsection{Subsection Title Goes Here}

Do you need subsections? Easy. You could even use \verb!\subsubsection{}! if you really want.

\subsection*{Subsection Without Numbering}

Want to include a picture? Remove the \verb!\begin{verbatim}! and \verb!\end{verbatim}! commands below, and change \verb!filename.jpg! to your filename. Save your graphs as image files and add them to your report using this method. Note: you won't see these commands in the PDF version. Verbatim changes the font and then writes everything while ignoring any formatting commands other than \verb!\end{verbatim}!. The inline version of verbatim is \verb!\verb!. If you read the .tex document, you'll see I've been using the \verb!\verb! command throughout.
\begin{verbatim}
\begin{center}
\includegraphics[width=0.35\textwidth]{filename.jpg}
\caption{Your caption goes here}
\end{center}
\end{verbatim}

Here's a table:
\begin{table}[h!]
\begin{center}
	\begin{tabular}{| c | r | c | c |}
		\hline
		Section                    & Due Date    & Weight 1 & Weight 2\\ \hline
		Lab 1: Period vs. Angle and Q Factor  & 30 Sep 2022 & 6\%  & 3\%  \\
		Lab 2: Period vs. Length and Q Factor vs. Length & 28 Oct 2022 & 6\%  & 3\%  \\
		Final Report & 2 Dec 2022 & 10\% & 16\% \\ \hline
	\end{tabular}
\caption{\label{TableName} Here's where you put the caption.}
\end{center}
\end{table}

\vspace{-0.5cm}
It usually puts a lot of space under tables. If you want less, you can use the \verb!\vspace{}! command. Negative numbers remove space, positive add space. Similarly, you can use \verb!\hspace{}! if you want horizontal space. This is rarely needed.


\section{Conclusion}

In conclusion, \LaTeX ~is pretty cool.

\appendix

\section{Some More Notes on TeX}

{\bf Here's how you make something bold.} \emph{Here's a way to make something in italics.}

You can make words {\large large} {\small or small} (or even {\huge huge} or {\tiny tiny}!).

Note that the \% sign makes things a comment. If you want to actually type the \% sign, you need to precede it with a
\textbackslash symbol like \verb!\%!.

If you're typing quotation marks, you might notice that typing them normally like you would on Word will "give you something like this." To get the opening quotation marks facing the right way, use backtick symbol \verb!``! instead of the quotation mark symbol \verb!"! for the opening quotes.

Here's a numbered list of things I hate:
\begin{enumerate}
\item Redundancy
		\item Misalignment (TeX fixes it though)
\item Irony
\item Numbered lists
\end{enumerate}
Let's repeat this with bullet points:
\begin{itemize}
\item Redundancy
		\item Misalignment 
\item Irony
\item Numbered lists
\end{itemize}




\end{document}


